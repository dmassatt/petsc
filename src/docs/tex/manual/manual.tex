%
%
\documentclass[twoside,11pt]{../sty/report_petsc}

\usepackage{makeidx,xspace}
\usepackage[bookmarksopen,colorlinks]{hyperref}
\usepackage[all]{hypcap}
\usepackage{color}
\input pdfcolor.tex

\usepackage[pdftex]{graphicx}


\usepackage{times}
\usepackage{listings}
\usepackage{tikz}
%\usepackage{psfig}
\usepackage{../sty/verbatim}
\usepackage{../sty/tpage}
\usepackage{../sty/here}
\usepackage{../sty/anlhelper}
\usepackage[hyphens,spaces,obeyspaces]{../sty/trl}

\setlength{\textwidth}{6.5in}
\setlength{\oddsidemargin}{0.0in}
\setlength{\evensidemargin}{0.0in}
\setlength{\textheight}{9.2in}
\setlength{\topmargin}{-.8in}

\newcommand{\findex}[1]{\index{#1}}
\newcommand{\sindex}[1]{\index{#1}}
\newcommand{\A}{\mbox{\boldmath \(A\)}}
\newcommand{\F}{\mbox{\boldmath \(F\)}}
\newcommand{\J}{\mbox{\boldmath \(J\)}}
\newcommand{\x}{\mbox{\boldmath \(x\)}}
\newcommand{\bb}{\mbox{\boldmath \(b\)}}
\newcommand{\rr}{\mbox{\boldmath \(r\)}}
hyperbaseurl

\makeindex

% Defines the environment where design issues are discussed. In the manual
% version of this report, these regions are ignored.
\def\design{\medskip \noindent Design Issue:\begin{em}}
\def\enddesign{\end{em} \medskip}
% Manual version:
% \def\design{\comment}
% \def\enddesign{\endcomment}

% Print DRAFT in large letters across every page
%\special{!userdict begin /bop-hook{gsave 200 70 translate
%65 rotate /Times-Roman findfont 216 scalefont setfont
%0 0 moveto 0.95 setgray (DRAFT) show grestore}def end}

% Defines that we're doing the whole manual, not the short intro part,
% used in part1.tex.
\def\shortintro{false}

\usepackage{fancyhdr,lastpage}
\pagestyle{fancy}
\rhead{PETSc 3.4 \today}

\begin{document}

%\pagestyle{empty}
%\begin{figure*}[hbt]
%\centerline{\includegraphics{titlepage1}}
%\end{figure*}

%  ANL changed the style of its title pages - so we are using titlepage1.pdf [above]
%  The following is an attempt to emulate titlepage1.pdf in latex

%%%%%%%%%%%%%%%%%%%%%%%%%%%%%%%%%%%%%%%%%%%%%%%%%%%%%%%%%%%%%%%%%%%%%%%%%%%%%%%%%%%%

\hfill {\large{\bf ANL-95/11}}

\vspace*{3in}
\noindent {\huge{\bf PETSc Users Manual}}
\vspace*{8pt}
\hrule
\vspace*{8pt}
\noindent {\huge{\it Revision 3.4}}

\vspace*{1in}
\noindent by \\
S. Balay, M. Adams, J. Brown, P. Brune, K. Buschelman, V. Eijkhout, W. Gropp, D. Kaushik, \\
M. Knepley, L. Curfman McInnes, K. Rupp, B. Smith, and H. Zhang \\
Mathematics and Computer Science Division, Argonne National Laboratory

\vspace*{10pt}
\noindent May 2013

\vspace*{20pt}
\noindent This work was supported by the Office of Advanced Scientific Computing Research, \\
Office of Science, U.S. Department of Energy, under Contract DE-AC02-06CH11357.

%%%%%%%%%%%%%%%%%%%%%%%%%%%%%%%%%%%%%%%%%%%%%%%%%%%%%%%%%%%%%%%%%%%%%%%%%%%%%%%%%%%%

\begin{figure*}[hbt]
\centerline{\includegraphics{titlepage2}}
\caption{}
\end{figure*}


\cleardoublepage
%\pagestyle{plain}

\vspace{1in}
\date{\today}

% Abstract for users manual
\addcontentsline{toc}{chapter}{Abstract}
% Abstract for TAO Users Manual

\addcontentsline{toc}{chapter}{Preface}
\section*{Preface}

The Toolkit for Advanced Optimization (TAO) focuses on the development
of algorithms and software for the solution of large-scale optimization 
problems on high-performance architectures.  Areas of interest include 
unconstrained and bound-constrained optimization, nonlinear least squares 
problems, optimization problems with partial differential equation 
constraints, and variational inequalities and complementarity 
constraints.

The development of TAO was motivated by the scattered support for
parallel computations and the lack of reuse of external toolkits in
current optimization software.  Our aim is to produce high-quality 
optimization software for computing environments ranging from 
workstations and laptops to massively parallel high-performance 
architectures.  Our design decisions are strongly motivated by 
the challenges inherent in the use of large-scale distributed 
memory architectures and the reality of working with large, 
often poorly structured legacy codes for specific 
applications.

This manual describes the use of TAO 2.2.0.  Since TAO is still under 
development, changes in usage and calling sequences may occur.  TAO 
is fully supported; see the the web site \url{http://www.mcs.anl.gov/tao} 
for information on contacting the developers.

%%% Local Variables: 
%%% mode: latex
%%% TeX-master: "manual_tex"
%%% End: 



\cleardoublepage

\input{gettinginfo.tex}

\medskip \medskip

\cleardoublepage

% Acknowledgements for users manual
% Acknowledgements for TAO Users Manual
%
% These are also listed on the TAO homepage, so if you add something here
% add it to the home page also
%

\addcontentsline{toc}{chapter}{Acknowledgments}
\section*{Acknowledgments}

We especially thank Jorge Mor\'e for his leadership, vision, and effort on 
previous versions of TAO.  

TAO relies on PETSc for the linear algebra required to solve optimization
problems, and we have benefited from the PETSc team's experience, tools, 
software, 
and advice. In many ways, TAO is a natural outcome of the PETSc 
development.  

TAO has benefited from the work of various researchers who have provided 
solvers, test problems, and interfaces.  In particular, we acknowledge Lisa 
Grignon, Elizabeth Dolan, Boyana Norris, Gabriel Lopez-Calva, Yurii Zinchenko, 
Michael Gertz, Jarek Nieplocha, Limin Zhang, Manojkumar Krishnan, and Evan 
Gawlik.  We also thank all TAO users for their comments, bug reports, and 
encouragement.

The development of TAO is supported by the Office of Advanced Scientific 
Computing Research, Office of Science, U.S. Department of Energy, under 
Contract DE-AC02-06CH11357.  We also thank the Argonne Laboratory Computing 
Resource Center and the National Energy Research Scientific Computing Center 
for allowing us to test and run TAO applications on their machines.

%%% Local Variables: 
%%% mode: latex
%%% TeX-master: "manual_tex"
%%% End: 


% Blank page makes double sided printout look bettter.

\cleardoublepage

\tableofcontents

% --------------------------------------------------------------------
%                            PART 1
% --------------------------------------------------------------------
\cleardoublepage
\part{Introduction to PETSc}
\label{part_intro}
\cleardoublepage
\chapter{Getting Started}
\input{part1tmp.tex}

% --------------------------------------------------------------------
%                            PART 2
% --------------------------------------------------------------------
\cleardoublepage
\part{Programming with PETSc}
\label{part_usage}
\input{part2tmp.tex}


%------------------------------------------------------------------


\cleardoublepage
\bibliographystyle{plain}
\addtocounter{chapter}{1}
\addcontentsline{toc}{chapter}{Bibliography}
\label{sec:bib}
\bibliography{../petsc,../petscapp}

\pagestyle{empty}
\begin{figure*}[hbt]
\centerline{\includegraphics{endpage}}
\caption{}
\end{figure*}

\end{document}


